\documentclass[a4paper,11pt]{report}
\usepackage{amsmath,amsthm,amssymb}
\usepackage[T1,T2A]{fontenc}
\usepackage[utf8]{inputenc}
\usepackage[english,russian]{babel}

\title{Двунаправленная очередь}
\author{Дувакин Кирилл Борисович}

\begin{document}
\maketitle
\tableofcontents
\chapter{Структура}
Элементом двунаправленной очереди является узел, в котором хранится переменная data - собственно данные, и указатель на следующий узел и предыдущий.

\chapter{Push/pop}
Основыми операциями с двунаправленной очередь являются:
\begin{enumerate}
    \item PushBack - добавление в конец очереди.
    Текущий хвост будет указывать на добавленный и добавленный станет хвостом
    \item PushFront - добавление в начало очереди. Добавленный узел будет указывать на текущий головной элемент очереди и сам станет головным.
    \item PopBack - удаление из конца очереди. Копируется хвост, хвостом становится, хвостом становится предыдущий текущего хвоста и возвращается скопированный.
    \item PopFront - удаление из начала очереди.Копируется текущий головной, головным элементом очереди станет следующий после текущего головного и вернется скопированный. 
\end{enumerate}

\chapter{Оценки сложности}
Добавление и удаление элементов O(1). Т.к. структура представляет собой двусвязный список, то обращение по индексам отсутствует. Соответственно, поиск элемента происходит за O(n), т.к. надо нужно с головного элемента переходить к следующему и т.д. 
\end{document}
